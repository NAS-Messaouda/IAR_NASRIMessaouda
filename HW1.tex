\documentclass[12pt]{article}
\usepackage{graphicx}
\graphicspath{ {images/} }
\title{Word-level Deep Sign Language Recognition}
\begin{document}
 Adrar  University \linebreak
 Faculty of Science and Technology \linebreak
Department of Mathematics and Computer Science \linebreak
(Initiation to Research 1) 2020/2021 \linebreak
\includegraphics [width=70mm , scale=0.5]{ASL}
\section*{ Title of  the article }
 Word-level Deep Sign Language Recognition from Video: \linebreak
A New Large-scale Dataset and Methods Comparison
\section*{Intruduction}
Deaf : A deaf person is considered to be suffering from a hearing 
impairment to the extent of a hearing loss of up to 70 dB. \linebreak
Sign language : is a term given to the non-vocal means of communication used by people with special needs,whether audio . 
\section*{Aim}
Vision-based sign language recognition aims at 
helping the hearing-impaired people to communicate with others.
We collect word-level signs extensively in ASL so we can know exactly what the person is asking for.
\section*{Arguments}
  One of the most important reasons for to investigate the chosen problem , There are 72 million deaf people  worldwide, according to the World Federation of the Deaf.
\section*{Objectives}
 Sign language requires more advanced learning algorithms that contain a large data set. We have started using a few images. In our future work, we aspire to use word level annotations to facilitate automatic translation of signs at the sentence and story level.
\section*{Information about the writer}
Full name : NASRI Messaouda    
\linebreak Contact information : (nas.messaouda@gmail.com)
\end{document}
